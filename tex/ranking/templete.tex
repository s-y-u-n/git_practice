\documentclass{article}

\usepackage{amsmath}
\setlength{\parindent}{0pt}

\title{social ranking}
\author{Tamura Shuntaro}

\begin{document}

\maketitle

\section{README}

Social Ranking Theory に関するメモ

\section{論文1}
Some Axiomatic and Algorithmic Perspectives on the Social Ranking Problem
stefano先生らの、序数的ランキング系統のベースの論文

\subsection{Abstract}

\subsection{Introduction}
・ランキングの使用例:就活、大学入試、オリンピック、アメリカ大統領選挙、ネットワークの中心性のランキング、など


・問い:有限集合Nとその部分集合に対するランキングが与えられたとき、Nに対する「社会的」ランキングを導き出すことができるのか?


例:
$S\subseteq N = \{1,2,3\}$

$\{1,2,3\}\succeq \{3\}\succeq \{1,3\} \succeq \{2,3\}\succeq \{2\}\succeq \{1,2\}\succeq \{1\}\succeq \emptyset $

\begin{equation}
    \label{eq:example1}
    f(x) = ax^2 + bx + c
\end{equation}

\begin{equation}
    \label{eq:example2}
    \frac{d}{dx} \left( e^x \right) = e^x
\end{equation}

Equation \ref{eq:example1} is a quadratic equation, and Equation \ref{eq:example2} is the derivative of the exponential function.
問い:有限集合Nとその部分集合に対するランキングが与えられたとき、Nに対する「社会的」ランキングを導き出すことができるのか?
例:$S\subseteq N = \{1,2,3\}$

\begin{equation}
    \label{eq:example1}
    f(x) = ax^2 + bx + c
\end{equation}

\begin{equation}
    \label{eq:example2}
    \frac{d}{dx} \left( e^x \right) = e^x
\end{equation}

Equation \ref{eq:example1} is a quadratic equation, and Equation \ref{eq:example2} is the derivative of the exponential function.

\section{Conclusion}

Feel free to add more sections and equations as needed. Happy note-taking!

\end{document}